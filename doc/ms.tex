\documentclass[a4paper, 12pt]{article}
\usepackage{amsmath}
\usepackage{amssymb}
\usepackage[left=2cm, right=2cm, bottom=3cm, top=2cm]{geometry}
\usepackage{graphicx}
\usepackage[utf8]{inputenc}
\usepackage{microtype}
\usepackage{natbib}

\newcommand{\Celery}{{\em Celery}}
\newcommand{\Exponential}{\textnormal{Exponential}}
\newcommand{\given}{\,|\,}
\newcommand{\Laplace}{\textnormal{Laplace}}
\newcommand{\location}{\textnormal{location}}
\newcommand{\scale}{\textnormal{scale}}


\title{The~\Celery~ model specification}
%\author{Brendon J. Brewer}
\date{}

\begin{document}
\maketitle

%\abstract{\noindent Abstract}

% Need this after the abstract
\setlength{\parindent}{0pt}
\setlength{\parskip}{1em}

Let $\boldsymbol{y} = \{y_1, y_2, ..., y_n\}$ be the vector of measurements
taken at times $\boldsymbol{t} = \{t_1, t_2, ..., t_n\}$. Assume that the
observers have provided ``error bars''
$\boldsymbol{\sigma} = \{\sigma_1, \sigma_2, ..., \sigma_n\}$ along with the
measurements.

Let the unknown parameters be $M$, the number of oscillation modes,
$\boldsymbol{T} = \{T_1, T_2, ..., T_M\}$ their periods,
$\boldsymbol{A} = \{A_1, A_2, ..., A_M\}$ their amplitudes,
and $\boldsymbol{Q} = \{Q_1, Q_2, ..., Q_M\}$ their qualities.
The priors for these are
\begin{align}
\ln T_i &\sim \Laplace(\location=a_T, \scale=b_T) \\
A_i &\sim \Exponential(\scale=\mu_A) \\
\ln Q_i &\sim \Laplace(\location=a_Q, \scale=b_Q)
\end{align}
where some hyperparameters relating to periods, amplitudes, and qualities
have been introduced.

%\begin{thebibliography}{999}
%\end{thebibliography}

\end{document}

